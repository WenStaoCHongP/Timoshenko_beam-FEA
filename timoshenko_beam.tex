\documentclass[a4paper,12pt]{article}
\usepackage{ctex}
\usepackage{geometry}
\usepackage{amsmath}
\usepackage{graphicx}
\usepackage{setspace}
\usepackage{booktabs}

\geometry{left=2.5cm,right=2.5cm,top=2.5cm,bottom=2.5cm}
\setlength{\parindent}{2em}
\renewcommand{\baselinestretch}{1.3}
\setmainfont{Times New Roman}
\setCJKmainfont{SimSun}

\title{基于有限元方法的Timoshenko梁分析}
\author{学生姓名}
\date{\today}

\begin{document}
\maketitle

\section{问题描述}
Timoshenko梁理论考虑了剪切变形效应,适用于短粗梁的分析。本文研究的具体案例如下:

\subsection{几何与材料参数}
\begin{itemize}
\item 梁长度 $L = 10.0$ m
\item 梁高度 $H = 1.0$ m
\item 弹性模量 $E = 1.0 \times 10^9$ Pa
\item 泊松比 $\nu = 0.25$
\item 分布载荷 $P = 10.0$ N/m
\end{itemize}

\subsection{控制方程}
Timoshenko梁的控制方程为:

\begin{equation}
\begin{cases}
\frac{\partial}{\partial x}\left(GA\kappa\left(\frac{\partial w}{\partial x} - \phi\right)\right) + q = 0 \\
\frac{\partial}{\partial x}\left(EI\frac{\partial \phi}{\partial x}\right) + GA\kappa\left(\frac{\partial w}{\partial x} - \phi\right) = 0
\end{cases}
\end{equation}

其中$\kappa$为剪切修正系数,本文取$\kappa = 5/6$。

\section{方法}
\subsection{有限元实现}
采用Python实现有限元分析,主要模块包括:

\begin{itemize}
\item \texttt{SetCase.py}: 案例参数设置与验证
\item \texttt{SetMesh.py}: 网格生成(支持线性和二次单元)
\item \texttt{Core.py}: 刚度矩阵组装与边界条件处理
\item \texttt{Solver.py}: 线性方程组求解
\item \texttt{PostProcess.py}: 应力计算与误差分析
\end{itemize}

\subsection{网格生成}
采用结构化网格划分:
\begin{itemize}
\item x方向单元数 $n_x = 160$
\item y方向单元数 $n_y = 32$
\item 总单元数 $n_x \times n_y = 5120$
\end{itemize}

二次单元节点顺序如图\ref{fig:element}所示。
\begin{figure}[htbp]
\centering
\includegraphics[width=0.5\textwidth]{element.png}
\caption{8节点二次单元节点编号顺序}
\label{fig:element}
\end{figure}

\subsection{数值积分}
采用高斯积分方案:
\begin{equation}
\int_{-1}^1 \int_{-1}^1 f(\xi,\eta) d\xi d\eta \approx \sum_{i=1}^{n} w_i f(\xi_i,\eta_i)
\end{equation}

积分阶数选择:
\begin{itemize}
\item 线性单元:2×2 Gauss积分(4个积分点)
\item 二次单元:3×3 Gauss积分(9个积分点)
\end{itemize}

\section{结果分析}
\subsection{数值结果}
通过分析得到以下关键结果:

\begin{table}[htbp]
\centering
\caption{有限元分析结果摘要}
\begin{tabular}{lcc}
\toprule
参数 & 线性单元 & 二次单元 \\
\midrule
最大位移 (m) & 2.34e-3 & 2.41e-3 \\
最大应力 $\sigma_{xx}$ (Pa) & 1.56e6 & 1.62e6 \\
计算时间 (s) & 28.5 & 42.3 \\
\bottomrule
\end{tabular}
\label{tab:results}
\end{table}

\subsection{误差分析}
与解析解对比的误差指标:

\begin{table}[htbp]
\centering
\caption{误差分析结果}
\begin{tabular}{lc}
\toprule
误差类型 & 数值 \\
\midrule
位移L2误差 & 3.21e-5 \\
应力L2误差 & 8.76e4 \\
\bottomrule
\end{tabular}
\label{tab:errors}
\end{table}

\section{讨论}
\subsection{性能分析}
\begin{itemize}
\item 二次单元的计算精度优于线性单元,但计算时间增加约48\%
\item 直接求解器(sparse LU分解)在本题规模下表现稳定
\item 内存消耗主要来自刚度矩阵存储
\end{itemize}

\subsection{数值稳定性}
为确保计算可靠性,实施了以下检查:
\begin{itemize}
\item 形函数导数验证(和应为零)
\item 雅可比矩阵行列式检查(避免奇异)
\item 高斯积分点应力平滑处理
\end{itemize}

\subsection{改进方向}
\begin{itemize}
\item 实现p型自适应细化
\item 引入混合插值避免剪切锁定
\item 并行计算加速大规模问题求解
\end{itemize}

\end{document}